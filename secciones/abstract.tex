	El presente proyecto de mejora se centra en la implementación de una nueva estrategia en el Hospital Regional ``Elvia Carrillo Castro'', con el objetivo de optimizar la aplicación del paquete de prevención de infecciones del sitio quirúrgico (ISQ) y capacitar al personal para su correcta ejecución. Las ISQ representan uno de los eventos adversos más comunes en los entornos hospitalarios, constituyendo entre el 15 y el 30 por ciento de todas las infecciones intrahospitalarias y presentando una tasa de mortalidad del 0.6 al 1.9 por ciento. Cada ISQ incrementa en promedio 7 días la estancia hospitalaria, lo que eleva considerablemente los costos de atención.

En México, se introdujo el Manual para la Implementación de Paquetes de Acciones para Prevenir y Vigilar las Infecciones Asociadas a la Atención de la Salud (IAAS) con el propósito de unificar la calidad en la atención sanitaria y mejorar la seguridad del paciente. Según la Organización Mundial de la Salud (OMS), las IAAS son infecciones adquiridas por pacientes durante su tratamiento en centros sanitarios, que no estaban presentes ni en incubación al momento del ingreso. Entre las IAAS, se destacan las infecciones del torrente sanguíneo, neumonías asociadas a ventilador, infecciones de vías urinarias y las ISQ.

Los paquetes de intervención son conjuntos de estrategias que, al ser integradas, actúan de manera sinérgica para reducir las tasas de infección, cada una respaldada por evidencia científica de alto nivel. La correcta implementación de estos paquetes requiere que todas las intervenciones sean aplicadas conjuntamente.

En el Hospital Regional ``Elvia Carrillo Castro'', se ha detectado un incumplimiento generalizado en la ejecución de las intervenciones especificadas en el manual, principalmente por la falta de aplicación adecuada por parte del personal de enfermería y médicos. Este proyecto de mejora busca corregir estas deficiencias mediante la actualización de la información dirigida al personal de salud y la implementación de mecanismos rigurosos de verificación. La capacitación continua en infecciones del sitio quirúrgico permitirá al personal adquirir las competencias necesarias para reducir el riesgo de infecciones postquirúrgicas, mejorando así los protocolos de seguridad y la atención efectiva a los pacientes, con el fin de disminuir las tasas de complicaciones y garantizar una recuperación adecuada.
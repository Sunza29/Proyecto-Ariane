\documentclass[12 pt]{report}
\setcounter{tocdepth}{3} % para que ponga subsubsecciones en el indice
\usepackage[utf8]{inputenc}
\usepackage[spanish]{babel}
\usepackage{graphicx}
\usepackage{amsmath}
\usepackage{setspace}

\begin{document}

	\begin{titlepage}
		\begin{center}
			\includegraphics [scale=0.5]{UVA-logo.png} \\
			\vspace{15 pt}
			\LARGE{UNIVERSIDAD VISCAYA DE LAS AMÉRICAS} \\
			\vspace{20 pt}
			\LARGE{LICENCIATURA EN ENFERMERÍA}\\
			\vspace{30 pt}
			\large{MEJORA EN LA APLICACIÓN DEL PAQUETE PARA LA PREVENCIÓN DE INFECCIONES DEL SITIO QUIRÚRGICO EN EL HOSPITAL REGIONAL ISSSTE ``ELVIA CARRILLO PUERTO'' }\\
			\vspace{40 pt}
			\large{PRESENTA:}\\
			\vspace{30 pt}
			\large{nombre de alumnos}\\
			\vspace{40 pt}
			\large{MÉRIDA, YUCATÁN}\\
			\large{xx de xx de 2025}		
		\end{center}
	\end{titlepage}	
	
	\tableofcontents
	\thispagestyle{empty}
	
	\begin{abstract}
			El presente proyecto de mejora se centra en la implementación de una nueva estrategia en el Hospital Regional ``Elvia Carrillo Castro'', con el objetivo de optimizar la aplicación del paquete de prevención de infecciones del sitio quirúrgico (ISQ) y capacitar al personal para su correcta ejecución.
	\end{abstract}
	
	
	\section*{Abreviaturas} \addcontentsline{toc}{section}{Abreviaturas}
	\section*{Antecedentes} \addcontentsline{toc}{section}{Antecedentes}
	\input{secciones/antecedentes}
	\section*{Justificación} \addcontentsline{toc}{section}{Justificación}
	\section*{Objetivos} \addcontentsline{toc}{section}{Objetivos}
	
	\subsection*{Objetivo general} \addcontentsline{toc}{subsection}{Objetivo general}
Mejorar la aplicación de los paquetes preventivos para la infección del sitio quirúrgico en el Hospital Elvia Carrillo Puerto, con el propósito de incrementar la seguridad del paciente y optimizar la calidad de la atención médica y de enfermería. Esto se logrará mediante la capacitación continua del personal de salud y el fortalecimiento de las medidas preventivas, asegurando una correcta implementación de las intervenciones basadas en evidencia científica.
	
	\subsubsection*{Objetivos específicos} \addcontentsline{toc}{subsubsection}{Objetivos específicos}
	\begin{itemize}
		\item Fortalecer la comunicación efectiva entre el personal médico y de enfermería para asegurar una correcta implementación de los paquetes preventivos y mejorar la coordinación en la atención al paciente.
		\item Implementar el uso de recursos informáticos y tecnológicos que faciliten el seguimiento y verificación de las medidas preventivas aplicadas en el ámbito quirúrgico.
		\item Promover y garantizar la seguridad del paciente sometido a intervención quirúrgica mediante la actualización y capacitación continua del personal de salud en la aplicación de las intervenciones basadas en evidencia científica.	
	\end{itemize}
	
	\section*{Materiales y métodos} \addcontentsline{toc}{section}{Materiales y métodos}
	
	\section*{Resultados} \addcontentsline{toc}{section}{Resultados}
	
	\section*{Discusión} \addcontentsline{toc}{section}{Discusión}
	
	\section*{Conclusión} \addcontentsline{toc}{section}{Conclusión}
	
	\section*{Anexos} \addcontentsline{toc}{section}{Anexos}


	\bibliographystyle{apalike}
	\bibliography{secciones/biblio}
	\addcontentsline{toc}{section}{Bibliografía}
	
	
	
	
\end{document}
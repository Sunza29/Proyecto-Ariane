\documentclass[12pt,a4paper]{article}
\usepackage[margin=1in,left=1.5in,top=1.5in,includefoot]{geometry}
\usepackage{fancyhdr}
\begin{document}
\pagenumbering{roman}
\begin{center}
\section*{ABSTRACT}
\addcontentsline{toc}{section}{\numberline{}Abstract}
\end{center}
The project aims at addressing the issues faced by the farmers in plucking arecanuts and spraying pesticides in the tree. These issues can be solved by a machine which can climb a tree without(automatic) or with (semi-automatic) the help of a man.
A tree climbing machine which can climb the tree automatically and spray pesticides with the help of a man who will control the sprayer setup on the ground has been developed. This machine needs a battery for its operation and can climb tree in approximately 10 feet per minute.
\newpage
\begin{center}
\section*{ACKNOWLEDGMENT}
\addcontentsline{toc}{section}{\numberline{}Acknowledgment}
\end{center}
It is our esteemed pleasure to present a project on “Automated Arecanut Tree Climbing Machine” Using Microcontroller and Machine Hardware setup.\\

Any Achievement big or small should have a catalyst and constant encouragement and advice of valuable and noble minds for our efforts to bring out this project. The satisfaction that accompanies the successful completion of any task would be incomplete without mentioning those who made it possible because success is the epitome of hard work, determination and dedication.\\

We express our deep gratitude to our project guide, Mr.Vayusutha M, Who gave us inspiration to pursue the project and guided us in this Endeavour.\\

We also express our deep gratitude to our project co-ordinator Mr. Dayanand GK for his support and guidance.\\

Our profound sense of gratitude is due to our Head of the department Prof. Subramanya Bhat for constant encouragement, Valuable guidance and also for providing us necessary facilities to carry out project successfully.\\

We like to thank our beloved Principal, Dr. Ganesh V Bhat and the management of \textbf{Canara Engineering College}.\\

Last but not the least we are thankful to all faculty members and lab instructors without whose support at various stages, this project would not have materialized.\\
\textbf{\begin{flushright}
Ashwin Prabhu\\
Darshan Mesta\\
Ganesha B\\
Harish Bhat
\end{flushright}}
\newpage
\tableofcontents
\addcontentsline{toc}{section}{\numberline{}Table of Content}
\newpage
\listoftables
\addcontentsline{toc}{section}{\numberline{}List of Tables}
\newpage
\listoffigures
\addcontentsline{toc}{section}{\numberline{}List of Figures}
\newpage
\section{INTRODUCTION}
\pagenumbering{arabic}
\pagestyle{fancy}
\fancyhead[R]{Automatic Tree Climber}
\fancyfoot{}
\fancyfoot[L]{Dept. of EC, CEC, Mangaluru}
\fancyfoot[R]{\thepage}
\renewcommand{\footrulewidth}{1pt}
\hspace{0.5cm}The arecanut palm is the source of common chewing nut, popularly known as betel nut or Supari. In India it is extensively used by large sections of people and is very much linked with religious practices. India is the largest producer of arecanut and at the same time largest consumer also. Karnataka produces about 40\% of the total yield throughout India.\\


Arecanut is a tree which grows nearly 12 to 30m. So it is a very difficult and risky job to climb these trees. Here we proposed an idea of developing a machine which can be used to climb the tree automatically controlled by a single person from ground. Thus it does not require an expert to climb the tree.
\newpage
\subsection{About Fruit Rot(Koleroga or Mahalia)}
\hspace{0.5cm} Fruit rot (Koleroga or Mahalia) is one of the major disease that destroys the crop yield in arecanut cultivation which can only be prevented/controlled by periodically spraying pesticides during rainy seasons which requires highly skilled labours whose availability now a days are very much limited. As a result farmers face problems in preserving the crop. Hence proposed automated sprayer will provide and ultimate solution to the worries of farmers.
\end{document}